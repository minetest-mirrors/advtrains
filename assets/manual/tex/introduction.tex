\section*{Introduction}
\addcontentsline{toc}{section}{Introduction}
\advtrains{} is a mod with the goal of introducing realistic trains and rail equipments. It also has features allowing automated trains on a large scale, including interlocking and a few methods of scripting.

This manual is written for the specific version of \advtrains{} that the corresponding \LaTeX{} source is distributed with. It is divided into two parts, where the first part mainly documents the various features of \advtrains, and the second part focuses on the internals of \advtrains.

\subsection*{Acknowledgments}

A notable portion of the manual is influenced by various other guides and manuals for \advtrains. In particular:
\begin{itemize}
\item The format of the API documentation is influenced by R$^7$RS.
\item Section \ref{s:interlocking} is written with reference to orwell's interlocking guide and Blockhead's video explaining the three-station setup. The links to both sources can be found in that section.
\item Section \ref{s:luaatc} is a reformatted and restructured version of the LuaATC documentation, written by orwell and Maverick2797.
\item The manuals related to railway time are modifed from the railway time API documentation written by orwell. Part of the section is also taken from \texttt{advtrains\_line\_automation/railwaytime.lua}
\end{itemize}

\subsection*{Conventions}
\begin{itemize}
\item Unless otherwise specified, the unit of length is meter (or, more technically, nodes).
\item Unless otherwise specified, the unit of speed is meter per second.
\item Unless otherwise specified, the unit of acceleration is meter per second squared.
\item The \tikz \pic at (0,0) {tcb}; symbol stands for a TCB assigned to the track.
\item The \tikz \pic (0,0) {signal=red}; symbol stands for a TCB assigned to the track with a signal assigned to the TCB side opposite of that of the direction in which the TCB is facing (here: the signal is assigned to the left side of the TCB).
\item Arguments to functions and ATC commands are delimited using angle brackets \var{like this}.
\item Argument names are generally lowercase, but the first letter may be capitalized.
\item Lua string constants are written without quotation marks when the content does not resemble any numeric or symbolic constant and when used as enumerators or table indices.
\item Like in other parts of \advtrains, Lua code is indented with tabs.
\item An error situation that may or may not be reported are referred to as \textit{an error}. An error situation that is always reported is one where \textit{an error is signaled}.
\item Unless otherwise specified, the return value of any Lua API function is undefined.
\item The action of left-clicking a node with a trackworker is called \textit{rotation}. The action of right-clicking a node with a trackworker is called \textit{adjustment}.
\end{itemize}

%%% Local Variables:
%%% TeX-master: "a4manual"
%%% End: